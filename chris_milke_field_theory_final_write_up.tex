\documentclass{article}
\usepackage{graphicx}
\usepackage{amsmath}
\usepackage{slashed}

\begin{document}

\title{The Geometry of the Yang-Mill's Field}
\author{Christopher Milke}

\maketitle

\section{Introduction}
        We define the pure fermion field as the spin-half solutions to the Dirac Equation, and construct our lagrangian density as:

        $ \mathcal{L} =  i \bar{\Psi} \slashed\partial_\mu \Psi - m \bar{\Psi} \Psi   $

        From here, the usual next step is to demand invariance of the Lagrangian under local $U(1)$ (Gauge) tranformations. $U(1)$ gauge tranformations take the form: 

        $ \Psi(x) \rightarrow e^{-i \alpha(x)} \Psi(x) $ and $ \bar\Psi(x) \rightarrow e^{+i \alpha(x)} \bar\Psi(x) $

        As the partial derivative of $\Psi$ does not transform this way in the lagrangian density, we introduce a new term into the lagrangian, 
        $  \bar{\Psi} \gamma^\mu \Psi A_\mu(x) $, which cancels the non-gauge-invariant part of the derivative. The new term with $A_\mu(x)$ can then be lumped in with the derivative term, to create a so-called ``Covariant'' derivative,
        $ D_\mu \equiv \partial_\mu - i e A_\mu $

\section{The ``Vector'' Fermion Field}
        We begin our discussion of the gauge field with a look at the fermion field $\Psi(x)$. We can, naively, define the fermion field as: 
        $\Psi \equiv R(x) \cdot \mathbf{S}$

        Here, $R(x)$ is the amplitude of the field at a given space-time point, and $\mathbf{S}$ is the spin-one-half 4-vector of the field.
        In this sense, the fermion field is technically a vector, but only in the context of its four-component spin vector. The spin vector however, does not change with time or space, and so is effectively a label. It becomes relevant only during interactions with the spin components of other fields.



\section{Parallel Transport}
        TODO

\section{Field Strength Tensor from Wilson Loop}
        TODO


\end{document}
