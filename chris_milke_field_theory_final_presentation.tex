\documentclass[12pt]{beamer}
\usepackage{graphicx}
\usepackage{amsmath}
\usepackage{slashed}

\begin{document}

\title{The Geometry of the Yang-Mill's Field}
\author{Christopher Milke}

\frame{\titlepage}

\begin{frame}
\frametitle{Outline}
        \begin{itemize}
                \item Quick look at gauge invariance \vfill
                \item A look at parallel transport \vfill
                \item Parellel transporting the fermion field \vfill
        \end{itemize}
\end{frame}

\begin{frame}
\frametitle{The Gauge Invariant Approach}
        Local $U(1)$ gauge tranformations: 
        \vfill

        $ \Psi(x) \rightarrow e^{-i \theta(x)} \Psi(x) $ and $ \bar\Psi(x) \rightarrow e^{+i \alpha(x)} \bar\Psi(x) $
        \vfill

        Base Fermion Lagrangian:
        $ \mathcal{L} =  i \bar{\Psi} \slashed\partial_\mu \Psi - m \bar{\Psi} \Psi   $
        \vfill

        Demand local $U(1)$ (Gauge) invariance: 
        $ D_\mu \equiv \partial_\mu - i e A_\mu $
        \vfill
\end{frame}

\begin{frame}
\frametitle{A Geometric Approach}
        Interpret Covarient Derivative as method for parallel transport
        
        \begin{center}
        \includegraphics[scale=.4]{Parallel_Transport}
        \end{center}

\end{frame}

\begin{frame}
\frametitle{An Explicit Definition of $\Psi(x)$}
        $\Psi(x) \equiv \mathbf{S} \cdot R(x) \cdot e^{i \phi(x)}$ \vfill

        $\mathbf{S}$: spin-one-half 4-vector of the field \vfill

        $R(x)$: amplitude of field \vfill
        
        $e^{i \phi(x)}$: phase factor \vfill
\end{frame}

\begin{frame}
\frametitle{The Fermion Field as a Vector}
        Parallel Transport requires a vector, so

        Expand complex phase as two real vectors:
        \vfill

        $\Psi(x) = \mathbf{S} \cdot R(x) \cdot
        \begin{bmatrix} \alpha_1\\ \alpha_2 \end{bmatrix} $ 
\end{frame}

\begin{frame}
\frametitle{Geometric vs True Change}
        All Changes in Phase are geometric
        \vfill
        Only changes in amplitude are true changes to the ``vector''
\end{frame}

\begin{frame}
\frametitle{Parallel Transport of Fermion Field}
        Start with $ D_\mu \equiv \partial_\mu + \Gamma_\mu $

        Demand
        $D_\mu \Psi(x) = \mathbf{S} \cdot [\partial_\mu R(x) ] \cdot e^{i\phi(x)} $

        Then: \vfill
        
        $ (\partial_\mu + \Gamma_\mu) [  \mathbf{S} \cdot R(x) \cdot e^{i\phi(x)} ]
                = \mathbf{S} \cdot (\partial_\mu R(x) ) \cdot e^{i\phi(x)} + 
                \mathbf{S} \cdot  R(x)  \cdot (\partial_\mu e^{i\phi(x)}) +
                \mathbf{S} \cdot  R(x)  \cdot e^{i\phi(x)} \Gamma_\mu $

        \vfill
        Implies that $\Gamma_\mu = - i \partial_\mu \phi(x) $. 
        \vfill
        
        Inserting a factor of $e$, we return to the original gauge field,

        $A_\mu \equiv \frac{1}{e} \partial_\mu \phi(x)$ 
        
        \vfill
        $\therefore D_\mu \equiv \partial_\mu - i e A_\mu $.
\end{frame}

\begin{frame}
\frametitle{Deriving (sort of) the Gauge Field Tensor}
        Parallel transport around a box


\end{frame}

\begin{frame}
\frametitle{Sources}
        \begin{itemize}
                \item \underline{Quantum Field Theory}; \textit{Lewis H. Ryder}; 1985 \vfill
                \item \underline{Introduction to Quantum Field Theory}; \textit{Michael E. Peskin, Daniel V. Schroeder}; 1951 \vfill
                \item \underline{Quantum Field Theory}; \textit{Mark Srednecki}; 2007 \vfill
                \item \underline{Relativity, Gravitation and Cosmology 2E}; \textit{Ta-Pei Cheng}; 2010 \vfill
        \end{itemize}
\end{frame}

\end{document}
